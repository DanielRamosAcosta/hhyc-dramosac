\documentclass{article}
\usepackage[spanish]{babel}
\usepackage[utf8]{inputenc}
\usepackage[T1]{fontenc}
\usepackage{hyperref}
\usepackage{xcolor}
\usepackage{listings}
\usepackage{minted}
\usepackage{graphicx}
\usepackage{longtable}
\usepackage{multirow}
\hypersetup{
    colorlinks,
    linkcolor={red!50!black},
    citecolor={blue!50!black},
    urlcolor={blue!80!black}
}

\title{P3: Rendimiento web}
\author{Daniel Ramos}
\date{\today}

\begin{document}

\maketitle

\begin{center}
    \large Herramientas HTML y CSS I
\end{center}

\newpage

\tableofcontents

\newpage

\section*{Introducción}

TODO

\begin{itemize}
    \item \href{https://github.com/DanielRamosAcosta/hhyc-dramosac}{Repositorio en GitHub}
    \item \href{https://www.danielramos.me/hhyc-dramosac}{Página web desplegada}
\end{itemize}

\newpage

\section{Análisis del tiempo de carga}\label{sec:analisis-de-tiempo-de-carga}

TODO

\subsection{Primer análisis (estado inicial)}\label{subsec:primer-analisis}

TODO

\begin{longtable}{c|c|c|c|c|c}
    \hline
    \textbf{TI} & \textbf{UR} & \textbf{TC} & \textbf{PT} & \textbf{PR} & \textbf{CR} \\
    \endhead
    \hline
    Inicio & \href{https://www.danielramos.me/hhyc-dramosac/index.html}{index.html} & 34,81s & 5812,8 Kb & 4951,36 Kb & 30 \\
    Recetas & \href{https://www.danielramos.me/hhyc-dramosac/recipes.html}{recipes.html} & 7,12s & 796 Kb & 618 Kb & 27 \\
    Escaldón & \href{https://www.danielramos.me/hhyc-dramosac/recipes-escaldon.html}{recipes-escaldon.html} & 20,7s & 4843,52 Kb & 2007,04 Kb & 46 \\
    Batata & \href{https://www.danielramos.me/hhyc-dramosac/recipes-batata.html}{recipes-batata.html} & 17,54s & 4485,12 Kb & 1648,64 Kb & 47 \\
    Enlaces & \href{https://www.danielramos.me/hhyc-dramosac/links.html}{links.html} & 631,2ms & 74 Kb & 50 Kb & 14 \\
    \hline
     \\[1.5ex]
     \caption{
          Ejemplos de imágenes optimizadas.
          Leyenda: 
          \textbf{TI}: Título de la página, 
          \textbf{UR}: URL, 
          \textbf{TC}: Tiempo de carga (promedio) 
          \textbf{PT}: Peso total 
          \textbf{PR}: Peso transferido, 
          \textbf{CR}: Cantidad de recursos que contine la página.
     }
    \label{tab:imagenes-optimizadas}
\end{longtable}

\subsection{Segundo análisis (tras aplicar \textit{lazy loading} y \texttt{defer})}\label{sec:segundo-analisis}

\begin{longtable}{c|c|c|c|c|c}
    \hline
    \textbf{TI} & \textbf{UR} & \textbf{TC} & \textbf{PT} & \textbf{PR} & \textbf{CR} \\
    \endhead
    \hline
    Inicio & \href{https://www.danielramos.me/hhyc-dramosac/index.html}{index.html} & 28,43s & 4812,8 Kb & 4751,36 Kb & 30 \\
    Recetas & \href{https://www.danielramos.me/hhyc-dramosac/recipes.html}{recipes.html} & 5,43s & 555 Kb & 541 Kb & 27 \\
    Escaldón & \href{https://www.danielramos.me/hhyc-dramosac/recipes-escaldon.html}{recipes-escaldon.html} & 16,1s & 4362,24 Kb & 1525,76 Kb & 46 \\
    Batata & \href{https://www.danielramos.me/hhyc-dramosac/recipes-batata.html}{recipes-batata.html} & 14,41s & 4167,68 Kb & 1331,2 Kb & 47 \\
    Enlaces & \href{https://www.danielramos.me/hhyc-dramosac/links.html}{links.html} & 527ms & 51,9 Kb & 43,72 Kb & 14 \\
    \hline
     \\[1.5ex]
     \caption{
          Ejemplos de imágenes optimizadas.
          Leyenda: 
          \textbf{TI}: Título de la página, 
          \textbf{UR}: URL, 
          \textbf{TC}: Tiempo de carga (promedio) 
          \textbf{PT}: Peso total 
          \textbf{PR}: Peso transferido, 
          \textbf{CR}: Cantidad de recursos que contine la página.
     }
    \label{tab:imagenes-optimizadas}
\end{longtable}

\subsection{Tercer análisis (tras aplicar mejoras propuestas)}\label{sec:tercer-analisis}

\begin{longtable}{c|c|c|c|c|c}
    \hline
    \textbf{TI} & \textbf{UR} & \textbf{TC} & \textbf{PT} & \textbf{PR} & \textbf{CR} \\
    \endhead
    \hline
    Inicio & \href{https://www.danielramos.me/hhyc-dramosac/index.html}{index.html} & 25,5s & 2396,16 Kb & 2293,76 Kb & 30 \\
    Recetas & \href{https://www.danielramos.me/hhyc-dramosac/recipes.html}{recipes.html} & 4,42s & 482 Kb & 455 Kb & 27 \\
    Escaldón & \href{https://www.danielramos.me/hhyc-dramosac/recipes-escaldon.html}{recipes-escaldon.html} & 2,3s & 223 Kb & 199 Kb & 46 \\
    Batata & \href{https://www.danielramos.me/hhyc-dramosac/recipes-batata.html}{recipes-batata.html} & 1,6s & 175 Kb & 151 Kb & 47 \\
    Enlaces & \href{https://www.danielramos.me/hhyc-dramosac/links.html}{links.html} & 623ms & 62 Kb & 42 Kb & 14 \\
    \hline
     \\[1.5ex]
     \caption{
          Ejemplos de imágenes optimizadas.
          Leyenda: 
          \textbf{TI}: Título de la página, 
          \textbf{UR}: URL, 
          \textbf{TC}: Tiempo de carga (promedio) 
          \textbf{PT}: Peso total 
          \textbf{PR}: Peso transferido, 
          \textbf{CR}: Cantidad de recursos que contine la página.
     }
    \label{tab:imagenes-optimizadas}
\end{longtable}

\subsection{Comparativa y análisis de resultados}\label{sec:comparativa-y-analisis-de-resultados}

\section{Primeros cambios}\label{sec:primeros-cambios}

\subsection{Aplicación de \texttt{loading="lazy"} en imágenes}\label{subsec:loading-lazy}

\subsection{Uso de \texttt{defer} en scripts}\label{subsec:defer-en-scripts}

\subsection{Justificación de los elementos modificados}\label{subsec:justificacion}

\section{Informe de mejoras}\label{sec:informe-de-mejoras}

\subsection{Resultados del test inicial (PageSpeed Insights)}\label{subsec:resultados-del-test-inicial}

\subsection{Cambios realizados en base a sugerencias}\label{subsec:cambios-realizados}

\subsection{Resultados tras las mejoras}\label{subsec:resultados-tras-las-mejoras}

\subsection{Análisis comparativo del rendimiento}\label{subsec:analisis-comparativo}

\section{Preguntas finales}\label{sec:preguntas-finales}

\subsection{Efecto de lazy loading en herramientas de desarrollo y en el rendimiento}\label{subsec:efecto-de-lazy-loading}

\subsection{Impacto y posibles problemas del uso de \texttt{defer}}\label{subsec:impacto-y-problemas-de-defer}

\subsection{Carga asíncrona de estilos: ¿es viable?}\label{subsec:carga-asincrona-de-estilos}

\end{document}
