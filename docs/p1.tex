\documentclass{article}
\usepackage[spanish]{babel}
\usepackage[utf8]{inputenc}
\usepackage[T1]{fontenc}
\usepackage{hyperref}
\usepackage{xcolor}
\usepackage{listings}
\usepackage{minted}
\usepackage{graphicx}
\hypersetup{
    colorlinks,
    linkcolor={red!50!black},
    citecolor={blue!50!black},
    urlcolor={blue!80!black}
}

\title{P1: Desarrollo de una Web}
\author{Daniel Ramos}
\date{\today}

\begin{document}

\maketitle

\begin{center}
    \large Herramientas HTML y CSS I
\end{center}

\newpage

\tableofcontents

\newpage

\section*{Introducción}
En esta práctica, aprenderemos a configurar el entorno de desarrollo necesario para trabajar con HTML y CSS. Esto incluye la instalación de editores de texto, navegadores web y otras herramientas útiles.

\newpage

\section{Configurando el Entorno de Desarrollo}\label{sec:configurando-el-entorno-de-desarrollo}

\subsection{Inicialización de Parcel}\label{subsec:inicializacion-de-parcel}

El primer paso ha sido inicializar el repositorio en el que se hará el desarrollo de la práctica con \lstinline|git init| y \lstinline|npm init|.

Siguiendo la guía de la web de parcel, lo siguiente ha sido instalar Parcel como dependencia del proyecto con \lstinline|npm install --save-dev parcel|.

Una vez instalado Parcel, se ha probado su correcto funcionamiento creando un \textit{Hello World}. Una vez creado el fichero \lstinline|src/index.html| se ha ejecutado el comando \lstinline|npx parcel src/index.html| y se ha comprobado que la página web de abría correctamente en el navegador como se ve en la Figura~\ref{fig:hello-world}.

 \begin{figure}[h!]
     \centering
     \includegraphics[width=0.8\textwidth]{./img/hello-world}
     \caption{Captura de pantalla de un Hello World básico en HTML}
     \label{fig:hello-world}
 \end{figure}

Se han realizado las modificaciones necesarias para incluir una hoja de estilos y script de ejemplos para comprobar que Parcel las procesa correctamente.
Sin tener que refrescar el navegador, los cambios se han reflejado instantáneamente en la página web (ver Figura~\ref{fig:parcel}).

 \begin{figure}[h!]
     \centering
     \includegraphics[width=0.8\textwidth]{./img/hello-world-styled}
     \caption{Captura de pantalla de la página web con Parcel}
     \label{fig:parcel}
 \end{figure}

Por último, siguiendo la guía, se han ajustado los scripts de \lstinline|package.json| para que sea más fácil ejecutar Parcel.
Ahora, en lugar de tener que escribir \lstinline|npx parcel src/index.html|, se puede ejecutar \lstinline|npm run start| para iniciar el servidor de desarrollo, y \lstinline|npm run build| para compilar el proyecto para producción.

Siguiendo con el enunciado del Módulo 2, se han ajustado una última vez los scripts del \lstinline|package.json|.
Se han añadido modificaciones para que se limpien los directorios de caché y de producción antes de cada compilación tal y como se puede ver en la Figura \ref{fig:package-json}.

\begin{figure}[h!]
\begin{minted}{json}
{
  "name": "hhyc-dramosac",
  "source": "src/index.html",
  "scripts": {
    "parcel:dev": "parcel",
    "parcel:build": "parcel build",
    "clean": "rimraf dist .parcel-cache",
    "start": "npm-run-all clean parcel:dev",
    "build": "npm-run-all clean parcel:build",
    "build:docs": "tectonic -Z shell-escape docs/p1.tex"
  },
  "author": "Daniel Ramos Acosta",
  "license": "ISC",
  "devDependencies": {
    "npm-run-all": "^4.1.5",
    "parcel": "^2.13.3",
    "rimraf": "^6.0.1"
  }
}
\end{minted}
\caption{Versión final del \lstinline{package.json}}
\label{fig:package-json}
\end{figure}

\subsection{Configuración de Pre y Post Procesadores}\label{subsec:configuracion-de-pre-y-post-procesadores}



\end{document}
