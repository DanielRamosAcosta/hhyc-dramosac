\documentclass{article}
\usepackage[spanish]{babel}
\usepackage[utf8]{inputenc}
\usepackage[T1]{fontenc}
\usepackage{hyperref}
\usepackage{xcolor}
\usepackage{listings}
\usepackage{minted}
\usepackage{graphicx}
\hypersetup{
    colorlinks,
    linkcolor={red!50!black},
    citecolor={blue!50!black},
    urlcolor={blue!80!black}
}

\title{P2: Web y recursos multimedia}
\author{Daniel Ramos}
\date{\today}

\begin{document}

\maketitle

\begin{center}
    \large Herramientas HTML y CSS I
\end{center}

\newpage

\tableofcontents

\newpage

\section*{Introducción}

En este documento se detallan las técnicas de desarrollo web utilizadas en la realización de la práctica 2 de la asignatura Herramientas HTML y CSS I.

\begin{itemize}
    \item \href{https://github.com/DanielRamosAcosta/hhyc-dramosac}{Repositorio en GitHub}
    \item \href{https://www.danielramos.me/hhyc-dramosac}{Página web desplegada}
\end{itemize}

\newpage

\section{Formatos de imagen utilizados}\label{sec:formatos-de-imagen-utilizados}

\subsection{Tabla de imágenes optimizadas}\label{subsec:tabla-de-imagenes-optimizadas}

\section{Utilización de las técnicas de imagen responsive}\label{sec:utilizacion-de-las-tecnicas-de-imagen-responsive}

\section{Animación de elementos en CSS}\label{sec:animacion-de-elementos-en-css}

\section{Uso de clip-path}\label{sec:uso-de-clip-path}

\section{Semántica y accesibilidad}\label{sec:semantica-y-accesibilidad}

\end{document}
